\documentclass[lang=cn,11pt,a4paper,cite=authoryear]{elegantpaper}

\title{COVID-19疫情趋势预测研究}
\author{肖世莺 \and 张红丽 \and 成宏媛 \and 王瑜}
\date{}

\usepackage{array}
\newcommand{\ccr}[1]{\makecell{{\color{#1}\rule{1cm}{1cm}}}}

\begin{document}

\maketitle

\begin{abstract}
本文。。。。。。。
如果想要了解本文的相关数据和程序代码,请访问
\href{https://github.com/data-science-in-action/project-stat223}{Github::project-stat223}
。
\keywords{COVID-19,SEIR,LSTM,预测分析}
\end{abstract}

\section{引言}
COVID-19已构成全球性大流行,并已蔓延到世界上大多数国家和地区。通过了解某个地区确
诊病例的发展趋势,政府可以采取相应的政策以控制应对疫情。但是,单一的模型估计可能会得出有
偏的结果,不同数学模型产生的预测结果是不一致的。。。。。。。

\section{相关研究}

\subsection{人口增长预测模型}

\subsection{传染病模型}

\subsection{机器学习的数量预测应用}

\subsection{COVID-19的预测研究}

\section{研究方法}

\subsection{SEIR模型}

\subsection{LSTM模型}
LSTM模型(Long Short-Term Memory)是一种用于深度学习领域的递归神经网络(RNN)架构,可以
学习长期依赖信息。LSTM由~\cite{hochreiter1997lstm}提出,此后许多学者对其进行了改进,
使得LSTM在许多问题上得到了广泛的使用,并取得相当巨大的成功\citep{gers2000learning, 
graves2005bidirectional, graves2005framewise, schmidhuber2007training, 
bayer2009evolving, schaul2010pybrain, graves2013hybrid, bayer2014Learning}。

\section{实证分析}

\subsection{SEIR模型}

\subsection{LSTM模型}

\section{讨论}

\bibliography{report}

\end{document}
